%% LyX 2.4.2.1 created this file.  For more info, see https://www.lyx.org/.
%% Do not edit unless you really know what you are doing.
\documentclass[10pt,english]{article}
\usepackage[T1]{fontenc}
\usepackage[latin9]{inputenc}
\synctex=-1
\usepackage{amsmath}
\usepackage{amsthm}

\makeatletter
%%%%%%%%%%%%%%%%%%%%%%%%%%%%%% Textclass specific LaTeX commands.
\numberwithin{equation}{section}
\numberwithin{figure}{section}

%%%%%%%%%%%%%%%%%%%%%%%%%%%%%% User specified LaTeX commands.
\pdfoutput=1

\usepackage{amsmath, amsthm, amssymb, amsfonts, standard}
\usepackage{bbm}
\usepackage{bbold}
\usepackage[basic]{complexity}
\usepackage{enumitem}
\usepackage{color}
\usepackage{fancyhdr}
\usepackage{algorithmic}
\usepackage[ruled]{algorithm}
\usepackage{fullpage}

\definecolor{shadecolor}{gray}{0.95}

\makeatother

\usepackage{babel}
\begin{document}
\global\long\def\R{\mathbb{R}}%
 
\global\long\def\Z{\mathbb{Z}}%
 
\global\long\def\C{\mathbb{C}}%
\global\long\def\Q{\mathbb{Q}}%

\global\long\def\ellOne{\ell_{1}}%
 
\global\long\def\ellTwo{\ell_{2}}%
 
\global\long\def\ellInf{\ell_{\infty}}%
 

\global\long\def\boldVar#1{\mathbf{#1}}%
 
\global\long\def\mvar#1{\boldVar{#1}}%
 
\global\long\def\vvar#1{\vec{#1}}%
 

\global\long\def\defeq{\stackrel{\mathrm{{\scriptscriptstyle def}}}{=}}%

\global\long\def\E{\mathbb{E}}%
\global\long\def\dist{\mathcal{D}}%

\global\long\def\otilde{\tilde{O}}%

\global\long\def\indicVec#1{\onesVec_{#1}}%

\global\long\def\gradient{\bigtriangledown}%
 
\global\long\def\grad{\gradient}%
 
\global\long\def\hessian{\gradient^{2}}%
 
\global\long\def\hess{\hessian}%
 

\global\long\def\innerProduct#1#2{\big\langle#1 , #2 \big\rangle}%
 
\global\long\def\norm#1{\|#1\|}%
\global\long\def\normFull#1{\left\Vert #1\right\Vert }%
 

\global\long\def\opt{\mathrm{opt}}%

\global\long\def\vzero{\vvar 0}%
 
\global\long\def\vones{\vvar 1}%
\global\long\def\onesVec{\vvar 1}%

\global\long\def\ma{\mvar A}%
 
\global\long\def\mb{\mvar B}%
 
\global\long\def\mc{\mvar C}%
 
\global\long\def\md{\mvar D}%
\global\long\def\mE{\mvar E}%
 
\global\long\def\mf{\mvar F}%
 
\global\long\def\mg{\mvar G}%
 
\global\long\def\mh{\mvar H}%
\global\long\def\mI{\mvar I}%
\global\long\def\mK{\mvar K}%
 
\global\long\def\mL{\mvar L}%
\global\long\def\mm{\mvar M}%
 
\global\long\def\mn{\mvar N}%
\global\long\def\mproj{\mvar P}%
\global\long\def\mq{\mvar Q}%
 
\global\long\def\mr{\mvar R}%
 
\global\long\def\ms{\mvar S}%
 
\global\long\def\mt{\mvar T}%
 
\global\long\def\mU{\mvar U}%
 
\global\long\def\mv{\mvar V}%
 
\global\long\def\mw{\mvar W}%
 
\global\long\def\mx{\mvar X}%
 
\global\long\def\my{\mvar Y}%
\global\long\def\mz{\mvar Z}%
 

\global\long\def\mSigma{\mvar{\Sigma}}%
 
\global\long\def\mLambda{\mvar{\Lambda}}%
 
\global\long\def\mpi{\mvar{\mathcal{\Pi}}}%

\global\long\def\mzero{\mvar 0}%
\global\long\def\mlap{\mvar{\mathcal{L}}}%

\global\long\def\mdiag{\mvar{diag}}%
\global\long\def\diag{\mathrm{diag}}%

\global\long\def\oracle{\mathcal{O}}%
 
\global\long\def\moracle{\mvar O}%
 
\global\long\def\oracleOf#1{\oracle\left(#1\right)}%
 

\global\long\def\nSamples{s}%
 
\global\long\def\simplex{\Delta}%

\global\long\def\abs#1{\left|#1\right|}%
\global\long\def\tr{\mathrm{tr}}%

\global\long\def\timeNearlyOp{\tilde{\mathcal{O}}}%
 
\global\long\def\timeNearlyLinear{\timeNearlyOp}%

\global\long\def\ceil#1{\left\lceil #1 \right\rceil }%

\global\long\def\runtime{\mathcal{T}}%
 
\global\long\def\timeOf#1{\runtime\left(#1\right)}%

\global\long\def\domain{\mathcal{D}}%

\global\long\def\argmin{\mathrm{argmin}}%
\global\long\def\argmax{\mathrm{argmax}}%

\global\long\def\nnz{\mathrm{nnz}}%
\global\long\def\vol{\mathrm{vol}}%
\global\long\def\supp{\mathrm{supp}}%

\global\long\def\energy{\xi}%
\global\long\def\indicDiff{\vec{\delta}}%
\global\long\def\congest{\mathrm{cong}}%
\global\long\def\poly{\mathrm{poly}}%
\global\long\def\congest{\mathrm{cong}}%

\global\long\def\boundary{\partial}%
\global\long\def\conductance{\mathrm{\phi}}%
\global\long\def\sparsity{\mathrm{\sigma}}%

\global\long\def\im{\mathrm{im}}%
\global\long\def\imbal{\mathrm{imbalance}}%


\title{\textbf{Combinatorial Optimization}\\
\textbf{Problem Set \#2 (Winter 2025)}}
\author{Instructor: Aaron Sidford (sidford@stanford.edu)\\
CA: Ta-Wei Tu (taweitu@stanford.edu)}
\date{}

\maketitle
This problem set is due Friday, January 24 at 5PM on Gradescope. Feel
free to discuss the problems with whoever you wish, but make sure
that the writing for your submission is entirely your own. As with
all assignments and exams in this course, unless stated otherwise,
you must prove your answers. You may use without proof anything proved
in the required reading (though you must clearly state what facts
you use).

Some of the problems may have \emph{hints}, which may be helpful for
solving the problem, \emph{remarks,} which give some extra context,
and \emph{bonuses}, which are optional, count for 0 points, and will
only be use to effect grades on the boundary (e.g., help determine
A+ grades), but are potentially interesting. It is possible to ignore
these hints, remarks, and bonuses and still obtaining a full score.

\textbf{\emph{These problem may ask you to prove something claimed
in the notes or in class. You cannot use those specific claims without
proof in your solution to the associated problems. :-)}}

\section*{Problem 1 {[}20 points{]}: Fast Cycle Decomposition}

For each of the problems below let $G=(V,E)$ be a simple, $n$-node,
$m$-edge unit capacity graph. Further, call $C=e_{1},...,e_{k}$
a \emph{simple cycle} if each $e_{i}=(a_{i},b_{i})$ with $a_{i+1}=b_{i}$
for all $i\in[k-1]$ and $b_{k}=a_{1}$ and the $a_{i}$ are distinct.
Note that $(a,b),(b,a)$ for $a\neq b$ is a simple cycle.\\
\\
\textbf{Part (a) {[}10 points{]}}: Prove that if each $a\in V$ has
out-degree at least 1, i.e., there exists $(a,b)\in E$, then $G$
has a simple cycle and there is an algorithm which provably finds
one in $O(n)$ time. You may assume that the input is given as the
adjacency list of the graph (i.e., a linked list of vertices where
each entry in the linked list holds a pointer to the list of edges
incident to the corresponding vertex).\\
\begin{shaded}
    test
\end{shaded}
\textbf{Part (b) {[}10 points{]}}: Let $f\in\{0,1\}^{E}$ be a circulation,
i.e., $f_{e}\in\{0,1\}$ for all $e\in E$, and $\im(f)_{v}=0$ for
all $v\in V$. Prove that there exist simple cycles $C_{1},...,C_{k}\subseteq E$
such that $f=\sum_{i\in[k]}c_{i}$ where $c_{i}=\indicVec{C_{i}}$
is the indicator vector for $C_{i}$, i.e.,
\[
[c_{i}]_{e}=[\indicVec{C_{i}}]_{e}=\begin{cases}
1 & \text{ if }e\in C_{i}\\
0 & \text{otherwise}
\end{cases}\,.
\]
\textbf{\emph{Bonus}}\emph{: Provide an algorithm that computes the
$C_{i}$} in part (b) in $O(|E|)$ time.\\
\textbf{\emph{}}\\
\textbf{\emph{Remark}}\emph{: Part (a) asks for a sublinear time algorithm
(whenever $m=\omega(n)$) for finding cycles in a graph. Part (b)
then builds upon this to prove that integer circulations can be decomposed
into simple cycles. Part (b) is essentially equivalent to proving
that Eulerian graphs, i.e., those where for every vertex its weighted
out-degree equals its weighted in-degree, can be decomposed into simple
cycles.}

\pagebreak{}

\section*{Problem 2 {[}25 Points{]}: Faster Dense Unit Capacity Maximum Flow}

For each of the following questions, consider the problem of computing
a $s$-$t$ maximum flow on a simple, $n$-node, $m$-edge unit capacity
graph $G=(V,E)$. Let $v_{*}$ denote the value of the maximum flow
in this instance and suppose $v_{*}>0$.\\
\\
\textbf{Part (a) {[}15 points{]}}: Prove that if $v_{*}>0$ then $v_{*}=O((n/d)^{2})$
where $d$ is the length of the shortest $s$-$t$ path in $G$.\textbf{\emph{}}\\
\textbf{\emph{}}\\
\textbf{Part (b) {[}10 points{]}}: Provide an algorithm that provably
computes an $s$-$t$ maximum flow in $O(mn^{2/3})$.\textbf{\emph{}}\\
\textbf{\emph{}}\\
\textbf{\emph{Bonus}}\emph{: Show that part (a) is un-improvable in
the worst case, in that for every integer $v_{*},d\geq6$ there is
a unit capacity graph with $O(v_{*}+d\sqrt{v_{*}})$ nodes such that
the value of the $s$-$t$ maximum flow is $v_{*}$ and the length
of the shortest path from $s$ to $t$ is $d$. }\\
\emph{}\\
\textbf{\emph{Remark}}\emph{: Together with what is proved in the
lecture notes, this shows that there is an $O(m\cdot\min\{m^{1/2},n^{2/3}\})$
time maximum flow algorithm for unit-capacity graph. This was the
state-of-the-art for almost 30 years until progress in the mid 2010s.}
\end{document}
