%% LyX 2.4.2.1 created this file.  For more info, see https://www.lyx.org/.
%% Do not edit unless you really know what you are doing.
\documentclass[10pt,english]{article}
\usepackage[T1]{fontenc}
\usepackage[latin9]{inputenc}
\synctex=-1
\usepackage{amsmath}
\usepackage{amsthm}

\makeatletter
%%%%%%%%%%%%%%%%%%%%%%%%%%%%%% Textclass specific LaTeX commands.
\numberwithin{equation}{section}
\numberwithin{figure}{section}

%%%%%%%%%%%%%%%%%%%%%%%%%%%%%% User specified LaTeX commands.
\pdfoutput=1

\usepackage{amsmath, amsthm, amssymb, amsfonts, standard}
\usepackage{bbm}
\usepackage{bbold}
\usepackage[basic]{complexity}
\usepackage{enumitem}
\usepackage{color}
\usepackage{fancyhdr}
\usepackage{algorithmic}
\usepackage[ruled]{algorithm}
\usepackage{fullpage}

\makeatother

\usepackage{babel}
\begin{document}
\global\long\def\R{\mathbb{R}}%
 
\global\long\def\Z{\mathbb{Z}}%
 
\global\long\def\C{\mathbb{C}}%
\global\long\def\Q{\mathbb{Q}}%

\global\long\def\ellOne{\ell_{1}}%
 
\global\long\def\ellTwo{\ell_{2}}%
 
\global\long\def\ellInf{\ell_{\infty}}%
 

\global\long\def\boldVar#1{\mathbf{#1}}%
 
\global\long\def\mvar#1{\boldVar{#1}}%
 
\global\long\def\vvar#1{\vec{#1}}%
 

\global\long\def\defeq{\stackrel{\mathrm{{\scriptscriptstyle def}}}{=}}%

\global\long\def\E{\mathbb{E}}%
\global\long\def\dist{\mathcal{D}}%

\global\long\def\otilde{\tilde{O}}%

\global\long\def\indicVec#1{\onesVec_{#1}}%

\global\long\def\gradient{\bigtriangledown}%
 
\global\long\def\grad{\gradient}%
 
\global\long\def\hessian{\gradient^{2}}%
 
\global\long\def\hess{\hessian}%
 

\global\long\def\innerProduct#1#2{\big\langle#1 , #2 \big\rangle}%
 
\global\long\def\norm#1{\|#1\|}%
\global\long\def\normFull#1{\left\Vert #1\right\Vert }%
 

\global\long\def\opt{\mathrm{opt}}%

\global\long\def\vzero{\vvar 0}%
 
\global\long\def\vones{\vvar 1}%
\global\long\def\onesVec{\vvar 1}%

\global\long\def\ma{\mvar A}%
 
\global\long\def\mb{\mvar B}%
 
\global\long\def\mc{\mvar C}%
 
\global\long\def\md{\mvar D}%
\global\long\def\mE{\mvar E}%
 
\global\long\def\mf{\mvar F}%
 
\global\long\def\mg{\mvar G}%
 
\global\long\def\mh{\mvar H}%
\global\long\def\mI{\mvar I}%
\global\long\def\mK{\mvar K}%
 
\global\long\def\mL{\mvar L}%
\global\long\def\mm{\mvar M}%
 
\global\long\def\mn{\mvar N}%
\global\long\def\mproj{\mvar P}%
\global\long\def\mq{\mvar Q}%
 
\global\long\def\mr{\mvar R}%
 
\global\long\def\ms{\mvar S}%
 
\global\long\def\mt{\mvar T}%
 
\global\long\def\mU{\mvar U}%
 
\global\long\def\mv{\mvar V}%
 
\global\long\def\mw{\mvar W}%
 
\global\long\def\mx{\mvar X}%
 
\global\long\def\my{\mvar Y}%
\global\long\def\mz{\mvar Z}%
 

\global\long\def\mSigma{\mvar{\Sigma}}%
 
\global\long\def\mLambda{\mvar{\Lambda}}%
 
\global\long\def\mpi{\mvar{\mathcal{\Pi}}}%

\global\long\def\mzero{\mvar 0}%
\global\long\def\mlap{\mvar{\mathcal{L}}}%

\global\long\def\mdiag{\mvar{diag}}%
\global\long\def\diag{\mathrm{diag}}%

\global\long\def\oracle{\mathcal{O}}%
 
\global\long\def\moracle{\mvar O}%
 
\global\long\def\oracleOf#1{\oracle\left(#1\right)}%
 

\global\long\def\nSamples{s}%
 
\global\long\def\simplex{\Delta}%

\global\long\def\abs#1{\left|#1\right|}%
\global\long\def\tr{\mathrm{tr}}%

\global\long\def\timeNearlyOp{\tilde{\mathcal{O}}}%
 
\global\long\def\timeNearlyLinear{\timeNearlyOp}%

\global\long\def\ceil#1{\left\lceil #1 \right\rceil }%

\global\long\def\runtime{\mathcal{T}}%
 
\global\long\def\timeOf#1{\runtime\left(#1\right)}%

\global\long\def\domain{\mathcal{D}}%

\global\long\def\argmin{\mathrm{argmin}}%
\global\long\def\argmax{\mathrm{argmax}}%

\global\long\def\nnz{\mathrm{nnz}}%
\global\long\def\vol{\mathrm{vol}}%
\global\long\def\supp{\mathrm{supp}}%

\global\long\def\energy{\xi}%
\global\long\def\indicDiff{\vec{\delta}}%
\global\long\def\congest{\mathrm{cong}}%
\global\long\def\poly{\mathrm{poly}}%
\global\long\def\congest{\mathrm{cong}}%

\global\long\def\boundary{\partial}%
\global\long\def\conductance{\mathrm{\phi}}%
\global\long\def\sparsity{\mathrm{\sigma}}%

\global\long\def\im{\mathrm{im}}%
\global\long\def\imbal{\mathrm{imbalance}}%


\title{\textbf{Combinatorial Optimization}\\
\textbf{Problem Set \#2 (Winter 2025)}}
\author{Instructor: Aaron Sidford (sidford@stanford.edu)\\
CA: Ta-Wei Tu (taweitu@stanford.edu)}
\date{}

\maketitle
This problem set is due Friday, January 24 at 5PM on Gradescope. Feel
free to discuss the problems with whoever you wish, but make sure
that the writing for your submission is entirely your own. As with
all assignments and exams in this course, unless stated otherwise,
you must prove your answers. You may use without proof anything proved
in the required reading (though you must clearly state what facts
you use).

Some of the problems may have \emph{hints}, which may be helpful for
solving the problem, \emph{remarks,} which give some extra context,
and \emph{bonuses}, which are optional, count for 0 points, and will
only be use to effect grades on the boundary (e.g., help determine
A+ grades), but are potentially interesting. It is possible to ignore
these hints, remarks, and bonuses and still obtaining a full score.

\textbf{\emph{These problem may ask you to prove something claimed
in the notes or in class. You cannot use those specific claims without
proof in your solution to the associated problems. :-)}}

\section*{Problem 1 {[}20 points{]}: Fast Cycle Decomposition}

For each of the problems below let $G=(V,E)$ be a simple, $n$-node,
$m$-edge unit capacity graph. Further, call $C=e_{1},...,e_{k}$
a \emph{simple cycle} if each $e_{i}=(a_{i},b_{i})$ with $a_{i+1}=b_{i}$
for all $i\in[k-1]$ and $b_{k}=a_{1}$ and the $a_{i}$ are distinct.
Note that $(a,b),(b,a)$ for $a\neq b$ is a simple cycle.\\
\\
\textbf{Part (a) {[}10 points{]}}: Prove that if each $a\in V$ has
out-degree at least 1, i.e., there exists $(a,b)\in E$, then $G$
has a simple cycle and there is an algorithm which provably finds
one in $O(n)$ time. You may assume that the input is given as the
adjacency list of the graph (i.e., a linked list of vertices where
each entry in the linked list holds a pointer to the list of edges
incident to the corresponding vertex).\\
\begin{shaded}
    We use the following naive idea. Pick a random start vertex $v_1$, and retrieve the list of vertices adjacent to $v_1$ (which is guaranteed to be nonempty). Take any edge $e_1 = (v_1,v_2) \in E$ and let $\tilde{V} = \{v_1,v_2\}$ store an unordered list of vertices and $\tilde{E} = (e_1)$ an ordered list of edges. Now get the list of adjacent vertices to $v_2$, and pick any vertex $v_3$ such that $e_2 = (v_2, v_3)$ is an edge. If $v_3 = v_i \in \tilde{V}$ for some $i < 3$ (which we can check in $O(1)$) return $C = (e_i, \ldots, e_2)$, otherwise add $v_3$ to $\tilde{V}$ and let $\tilde{E} = (e_1, e_2)$. Continue this process until $v_k = v_i \in \tilde{V}$ for some $i < k$.
    \begin{lstlisting}
    |\hrule \textbf{Simple Cycle Algorithm} \hrule|
input :Graph $G = (V,E)$, $V$ linked list of vertices.
        Each $v \in V$ points to list of edges incident to $v$.
        $G$ has out-degree at least $1$ at every vertex.
Let $v_1 \in V$
Let $\tilde{V} = \{v_1\}$
Let $\tilde{E} = ()$
for (k = 1; k < n; k++) :
    Let $I := $ list of edges pointed to by $v_k$
    Let $e_k := (v_k, v_{k+1}) = I[0]$
    if $v_{k+1} = v_i \in \tilde{V}$ :
        return $C = e_i \ldots e_{k}$
    else :
        $E = e_1 \ldots e_{k+1}$ // append $e_{k+1}$ to $E$
|\hrule|
    \end{lstlisting}
    This finds a cycle in $O(n)$. Since $V$ is finite, $\tilde{V}$ can hold at most $n$ vertices which means the for loop goes through at most $n$ iterations before returning, i.e. a value is always returned. The $C$ that is returned is always a cycle because each edge $(v_i, v_{i+1})$ is chosen such that the next edge $(v_{i+1}, v_{i+2})$ shared vertex $v_{i+1}$, i.e. it is a path, and if $v_i$ is the first vertex that has been visited before by $e_1 \ldots e_{i-1}$ then it must be $v_j \in \tilde{V}$ for some $1 \leq j < i$. Then $e_i = (v_i, v_{i+1}) = (v_i, v_j)$ connects to $e_j = (v_j, v_{j+1})$, and since this is the first such vertex every two edges in $e_j \ldots e_i$ are distinct, i.e. this is a cycle.
\end{shaded}
\textbf{Part (b) {[}10 points{]}}: Let $f\in\{0,1\}^{E}$ be a circulation,
i.e., $f_{e}\in\{0,1\}$ for all $e\in E$, and $\im(f)_{v}=0$ for
all $v\in V$. Prove that there exist simple cycles $C_{1},...,C_{k}\subseteq E$
such that $f=\sum_{i\in[k]}c_{i}$ where $c_{i}=\indicVec{C_{i}}$
is the indicator vector for $C_{i}$, i.e.,
\[
[c_{i}]_{e}=[\indicVec{C_{i}}]_{e}=\begin{cases}
1 & \text{ if }e\in C_{i}\\
0 & \text{otherwise}
\end{cases}\,.
\]
\begin{shaded}
    We continue by induction on the number of edges $m$. If $m = 1$ then this vacuously holds for no cycles since any circulation must have $f_e = 0$ on the unique edge. 
    
    Suppose this holds for all graphs with $\leq m - 1$ edges, and let $G$ be a graph with $m$ edges. Let $G'$ be $G$ with all weight $0$ edges from $G$ removed. If we removed at least one edge then by the inductive hypothesis there are cycles such that $f = \sum_{i=1}^k c_i$ on $G'$. Note this also then holds on $G$ since on weight $0$ edges, $f_e = 0 = \sum_{i=1}^k [c_i]_e$ as $C_i$ do not contain the edges $e$ we removed.

    Therefore, suppose $G$ has all $m$ edges with $f_e = 1$. Let $G_0$ be a connected component of $G$ with at least one edge. Note that $\im(f)_v = 0$ for every $v \in G_0$ since the edges incident to $v$ in $G$ and $G_0$ are the same. Therefore every vertex in $G_0$ must have at least one outgoing edge since otherwise it would just have incoming edges (by connectedness of $G_0$ it has at least $1$ edge) and hence nonzero imbalance.
    
    By part (a), we can find a simple cycle $C_1$ in $G_0$. Note that removing edges in a cycle does not change $\im(f)_v = 0$ for all $v \in G_0$ since every vertex in the cycle has exactly one incoming and one outgoing edge. The same $\im(f)_v =0$ also holds for vertices in $G$ not in $G_0$ since we don't change any edges incident to them when removing $C_1$ by connectedness of $G_0$. Therefore, by the inductive hypothesis on $(V, E \setminus C_1)$, there are disjoint simple cycles $C_2, \ldots, C_k$ such that $f = \sum_{i=2}^k c_i$ on edges in $E \setminus C_1$. We will show that $f = \sum_{i=1}^k c_i$ on all edges in $E$.

    If $e \in C_1$ then $[c_i]_e = 0$ for $i = 2,\ldots,k$ since we chose them as subsets of $E \setminus C_1$, so $f_e = [c_1]_e = 1$. If $e \not \in C_1$ then $[c_1]_e = 0$, and by the inductive hypothesis $\sum_{i=1}^k [c_i]_e = \sum_{i=2}^k [c_i]_e = f_e$. By induction on $m$, for any graph $G$ and circulation $f$ there exist simple cycles $C_1,\ldots,C_k \subseteq E$ such that $f = \sum_{i=1}^k c_i$.
\end{shaded}
\textbf{\emph{Bonus}}\emph{: Provide an algorithm that computes the
$C_{i}$} in part (b) in $O(|E|)$ time.\\
\textbf{\emph{}}\\
\textbf{\emph{Remark}}\emph{: Part (a) asks for a sublinear time algorithm
(whenever $m=\omega(n)$) for finding cycles in a graph. Part (b)
then builds upon this to prove that integer circulations can be decomposed
into simple cycles. Part (b) is essentially equivalent to proving
that Eulerian graphs, i.e., those where for every vertex its weighted
out-degree equals its weighted in-degree, can be decomposed into simple
cycles.}

\pagebreak{}

\section*{Problem 2 {[}25 Points{]}: Faster Dense Unit Capacity Maximum Flow}

For each of the following questions, consider the problem of computing
a $s$-$t$ maximum flow on a simple, $n$-node, $m$-edge unit capacity
graph $G=(V,E)$. Let $v_{*}$ denote the value of the maximum flow
in this instance and suppose $v_{*}>0$.\\
\\
\textbf{Part (a) {[}15 points{]}}: Prove that if $v_{*}>0$ then $v_{*}=O((n/d)^{2})$
where $d$ is the length of the shortest $s$-$t$ path in $G$.\textbf{\emph{}}\\
\textbf{\emph{}}\\
\begin{shaded}
    Let $L_i = \{v \in V \mid d(v) = i\}$ be the set of vectors of minimum distance at most $i$ from $s$. Let $S_i = \{(v,w) \in E \mid v \in L_{i-1}, w \in L_i\}$ be the set of edges $(v,w)$ where $v$ is minimum distance $i-1$ and $w$ is minimum distance $i$ from $s$. Each of $S_1, \ldots, S_d$ is an $s-t$ cut since $d$ is the minimum distance between $s$ and $t$ by definition.
    Let $\ell_i = |L_i|$ and note that $|S_i| \leq \ell_{i-1} \ell_i$ since our graph is simple, i.e. no repeated edges between two nodes. The $L_i$ are disjoint, and so $\ell_0 + \ldots + \ell_d \leq n$. Suppose that $\ell_{i-1} \ell_i \geq n^2/d^2$ for all $i = 1, \ldots, d$. Then by AM-GM, $\frac{\ell_{i-1} + \ell_i}{2} \geq n/d$ so by summing over $i = 1, \ldots, d$ we get
    \[\ell_0/2 + \ell_1 + \ldots + \ell_{d-1} + \ell_d/2 \geq n\]
    which implies $\ell_0 + \ldots + \ell_d > n$, a contradiction. Therefore, we must have that $\ell_{i-1}\ell_i < n^2/d^2$ for some $i = 1, \ldots, d$, which means $|S_i| < n^2/d^2$. Since the max flow $v_*$ is bounded above by the min cut, and $S_i$ is a cut of capacity $|S_i|$ since the graph is unit capacity, $v_* \leq |S_i| < n^2/d^2$.
\end{shaded}
\textbf{Part (b) {[}10 points{]}}: Provide an algorithm that provably
computes an $s$-$t$ maximum flow in $O(mn^{2/3})$.\textbf{\emph{}}\\
\textbf{\emph{}}\\
\begin{shaded}
We use the blocking flow algorithm, Algorithm 7.1 in the notes.
    \begin{lstlisting}
|\hrule| |\textbf{Max Flow Algorithm}| |\hrule|
input : unit capacitated simple graph and vertices $s,t \in V$ with $s \neq t$
Let $f_0 = 0 \in \mathbb{R}^E$ and $i = 0$
while $d_{f_i}(t) \neq \infty$ do :
    Compute blocking flow $r_i$ in $\call_{f_i}$
    Let $f_{i+1}$ := augment($f_i$, $r_i$)
    $i$ := $i + 1$
end
Return $f_i$
    \end{lstlisting}
By Theorem 7.7 we can compute a blocking flow in $O(m)$, and it always returns a maximal flow by Theorem 7.6 in the notes, so we need to bound the number of iterations of the loop by $O(n^{2/3})$ as in Theorem 7.9 using our bound from part (a). 

By Lemma 7.5 in the notes $d_{f_{i+1}}(t) \geq d_{f_i}(t)$ for all $i$, and since distances are positive this implies $d_{f_i}(t) \geq i$ for all $i$ in which the while loop has not yet terminated. Suppose the while loop has not terminated for $i = \lceil n^{2/3} \rceil$ such that $d_{f_i}(t) \geq \lceil n^{2/3} \rceil$. Note that $G_{f_i}$ is still a simple unit capacity graph since if $(v,w) \in E$ then $(w,v)$ is added to $E_{f_i}$ with capacity $[f_i]_{(v,w)}$ which is either $1$ or $0$ (and $(v,w) \in E_{f_i}$ has capacity $u_{(v,w)} - [f_i]_{(v,w)}$ which is $1$ or $0$), hence we still don't have multiple edges and the new edges are unit capacity. Further note that $d_{f_i}(t)$ is by definition the length of the shortest $s$-$t$ path (along nonzero capacity edges) in $G_{f_i}$. By part (a), $v_*(G_{f_i}) = O(n^2/d_{f_i}(t)^2) = O(n^2/n^{4/3}) = O(n^{2/3})$. Since each iteration of the algorithm must increase $v(f_i)$ by at least $1$ and therefore decrease $v_*(G_{f_i})$ by at least $1$, it must be that the algorithm terminates in an additional, at most, $O(n^{2/3})$ iterations and therefore the algorithm terminates in $O(mn^{2/3})$ time.
\end{shaded}
\textbf{\emph{Bonus}}\emph{: Show that part (a) is un-improvable in
the worst case, in that for every integer $v_{*},d\geq6$ there is
a unit capacity graph with $O(v_{*}+d\sqrt{v_{*}})$ nodes such that
the value of the $s$-$t$ maximum flow is $v_{*}$ and the length
of the shortest path from $s$ to $t$ is $d$. }\\
\emph{}\\
\textbf{\emph{Remark}}\emph{: Together with what is proved in the
lecture notes, this shows that there is an $O(m\cdot\min\{m^{1/2},n^{2/3}\})$
time maximum flow algorithm for unit-capacity graph. This was the
state-of-the-art for almost 30 years until progress in the mid 2010s.}
\end{document}
