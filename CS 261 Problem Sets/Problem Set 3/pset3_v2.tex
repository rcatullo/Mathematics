%% LyX 2.4.3 created this file.  For more info, see https://www.lyx.org/.
%% Do not edit unless you really know what you are doing.
\documentclass[10pt,english]{article}
\usepackage[T1]{fontenc}
\usepackage[latin9]{inputenc}
\synctex=-1
\usepackage{standard}

\makeatletter
%%%%%%%%%%%%%%%%%%%%%%%%%%%%%% Textclass specific LaTeX commands.
\numberwithin{equation}{section}
\numberwithin{figure}{section}

%%%%%%%%%%%%%%%%%%%%%%%%%%%%%% User specified LaTeX commands.
\pdfoutput=1


\usepackage{bbm}
\usepackage{bbold}
\usepackage[basic]{complexity}
\usepackage{algorithmic}
\usepackage[ruled]{algorithm}
\usepackage{fullpage}

\makeatother

\usepackage{babel}
\begin{document}
\global\long\def\R{\mathbb{R}}%
 
\global\long\def\Z{\mathbb{Z}}%
 
\global\long\def\C{\mathbb{C}}%
\global\long\def\Q{\mathbb{Q}}%

\global\long\def\ellOne{\ell_{1}}%
 
\global\long\def\ellTwo{\ell_{2}}%
 
\global\long\def\ellInf{\ell_{\infty}}%
 

\global\long\def\boldVar#1{\mathbf{#1}}%
 
\global\long\def\mvar#1{\boldVar{#1}}%
 
\global\long\def\vvar#1{\vec{#1}}%
 

\global\long\def\defeq{\stackrel{\mathrm{{\scriptscriptstyle def}}}{=}}%

\global\long\def\E{\mathbb{E}}%
\global\long\def\dist{\mathcal{D}}%

\global\long\def\otilde{\tilde{O}}%

\global\long\def\indicVec#1{\onesVec_{#1}}%

\global\long\def\gradient{\bigtriangledown}%
 
\global\long\def\grad{\gradient}%
 
\global\long\def\hessian{\gradient^{2}}%
 
\global\long\def\hess{\hessian}%
 

\global\long\def\innerProduct#1#2{\big\langle#1 , #2 \big\rangle}%
 
\global\long\def\norm#1{\|#1\|}%
\global\long\def\normFull#1{\left\Vert #1\right\Vert }%
 

\global\long\def\opt{\mathrm{opt}}%

\global\long\def\vzero{\vvar 0}%
 
\global\long\def\vones{\vvar 1}%
\global\long\def\onesVec{\vvar 1}%

\global\long\def\ma{\mvar A}%
 
\global\long\def\mb{\mvar B}%
 
\global\long\def\mc{\mvar C}%
 
\global\long\def\md{\mvar D}%
\global\long\def\mE{\mvar E}%
 
\global\long\def\mf{\mvar F}%
 
\global\long\def\mg{\mvar G}%
 
\global\long\def\mh{\mvar H}%
\global\long\def\mI{\mvar I}%
\global\long\def\mK{\mvar K}%
 
\global\long\def\mL{\mvar L}%
\global\long\def\mm{\mvar M}%
 
\global\long\def\mn{\mvar N}%
\global\long\def\mproj{\mvar P}%
\global\long\def\mq{\mvar Q}%
 
\global\long\def\mr{\mvar R}%
 
\global\long\def\ms{\mvar S}%
 
\global\long\def\mt{\mvar T}%
 
\global\long\def\mU{\mvar U}%
 
\global\long\def\mv{\mvar V}%
 
\global\long\def\mw{\mvar W}%
 
\global\long\def\mx{\mvar X}%
 
\global\long\def\my{\mvar Y}%
\global\long\def\mz{\mvar Z}%
 

\global\long\def\mSigma{\mvar{\Sigma}}%
 
\global\long\def\mLambda{\mvar{\Lambda}}%
 
\global\long\def\mpi{\mvar{\mathcal{\Pi}}}%

\global\long\def\mzero{\mvar 0}%
\global\long\def\mlap{\mvar{\mathcal{L}}}%

\global\long\def\mdiag{\mvar{diag}}%
\global\long\def\diag{\mathrm{diag}}%

\global\long\def\oracle{\mathcal{O}}%
 
\global\long\def\moracle{\mvar O}%
 
\global\long\def\oracleOf#1{\oracle\left(#1\right)}%
 

\global\long\def\nSamples{s}%
 
\global\long\def\simplex{\Delta}%

\global\long\def\abs#1{\left|#1\right|}%
\global\long\def\tr{\mathrm{tr}}%

\global\long\def\timeNearlyOp{\tilde{\mathcal{O}}}%
 
\global\long\def\timeNearlyLinear{\timeNearlyOp}%

\global\long\def\ceil#1{\left\lceil #1 \right\rceil }%

\global\long\def\runtime{\mathcal{T}}%
 
\global\long\def\timeOf#1{\runtime\left(#1\right)}%

\global\long\def\domain{\mathcal{D}}%

\global\long\def\argmin{\mathrm{argmin}}%
\global\long\def\argmax{\mathrm{argmax}}%

\global\long\def\nnz{\mathrm{nnz}}%
\global\long\def\vol{\mathrm{vol}}%
\global\long\def\supp{\mathrm{supp}}%

\global\long\def\energy{\xi}%
\global\long\def\indicDiff{\vec{\delta}}%
\global\long\def\congest{\mathrm{cong}}%
\global\long\def\poly{\mathrm{poly}}%
\global\long\def\congest{\mathrm{cong}}%

\global\long\def\boundary{\partial}%
\global\long\def\conductance{\mathrm{\phi}}%
\global\long\def\sparsity{\mathrm{\sigma}}%

\global\long\def\im{\mathrm{im}}%
\global\long\def\imbal{\mathrm{imbalance}}%


\title{\textbf{Combinatorial Optimization}\\
\textbf{Problem Set \#3 (Winter 2025)}}
\author{Instructor: Aaron Sidford (sidford@stanford.edu)\\
CA: Ta-Wei Tu (taweitu@stanford.edu)}
\date{}

\maketitle
This problem set is due Friday, January 31 at 5PM on Gradescope. Feel
free to discuss the problems with whoever you wish, but make sure
that the writing for your submission is entirely your own. As with
all assignments and exams in this course, unless stated otherwise,
you must prove your answers. You may use without proof anything proved
in the required reading (though you must clearly state what facts
you use).

Some of the problems may have \emph{hints}, which may be helpful for
solving the problem, \emph{remarks,} which give some extra context,
and \emph{bonuses}, which are optional, count for 0 points, and will
only be use to effect grades on the boundary (e.g., help determine
A+ grades), but are potentially interesting. It is possible to ignore
these hints, remarks, and bonuses and still obtaining a full score.

\section*{Problem 1 {[}10 Points{]}: Directing an Undirected Graph}

Let $G=(V,E)$ be an undirected graph and $\mu\in\Z_{>0}^{V}$. We
say that $G$ is \emph{$\mu$-orientable }if we can replace each undirected
$\{u,v\}\in E$ with one of $(u,v)\in E$ or $(v,u)\in E$ such that
that $\mathrm{deg}_{\mathrm{out}}(u)\defeq|\{(u,v)\in E\}|$ satisfies
$\mathrm{deg}_{\mathrm{out}}(u)\leq\mu(u)$ for each $u\in V$. Provide
a polynomial time algorithm that given $G$ and $\mu$ provably outputs
whether or not $G$ is $\mu$-orientable.

\begin{shaded}
SOLUTION: Construct a new directed graph $G'$ with vertices $s,t$, $v \in V$, and $e \in E$. That is, there are $m + n + 2$ vertices in the new graph with $n$ corresponding to the vertices in $G$, $m$ corresponding to the edges in $E$, and a source and sink vertex. Let our edges $E'$ be $(s, e)$ for every $e \in E$ of capacity $1$, and $(v, t)$ of capacity $\mu(v)$ for every $v \in V$. Give an edge $(e, v)$ of capacity $1$ if $v$ is incident to $e$, i.e. $e = \{v,u\}$ for some $u \in V$.

Then $G$ is $\mu$-orientable if and only if the max flow $v_*(G') = m$. Note that $\deg_{\text{out}}(s) = m$ so $v_*(G') \leq m$ always. Suppose $v_*(G') = m$ with max flow $f$. For each edge $e \in G$, $f_{(s,e)} = 1$ so the outgoing flow from vertex $e$ is $1$. If $e = \{v,u\}$ then $f_{(e,v)} = 1$ and $f_{(e,u)} = 0$ up to reordering of $u$ and $v$. In the new directed graph, we let $e = (v,u)$, i.e. if the flow of $(e,v)$ is $1$ in $G'$ then we orient the edge such that $v$ is the initial vertex of $e$. Then the outdegree of $v$ in this new oriented graph is at most $\mu(v)$. To see this, note that in $G'$ the capacity of $(v,t)$ is $\mu(v)$, which means at most $\mu(v)$ edges $e$ have $f_{(e,v)} = 1$. That is, at most $\mu(v)$ edges have $v$ as the initial vertex. It follows that $G$ is $\mu$-orientable.

Suppose $G$ is $\mu$-orientable. Then we follow the steps in reverse and construct a flow of value $m$. Send $1$ unit of flow along each $(s,e)$ in $G'$. If $e = \{v,u\}$ is oriented with $v$ as its initial vertex in the $\mu$-oriented graph, let $f_{(e,v)} = 1$ and $f_{(e,u)} = 0$. Finally, let $f_{(v,t)} = \deg_{\text{out}}(v) \leq \mu(v)$ by assumption, so the flow imbalance across $v$ in $G'$ is $0$ and the capacity of $(v,t)$ is respected. It follows that $f$ defines a flow of value $m$ since all interior vertices are balanced and $m$ units of flow are sent out of $s$. 

Ford-Fulkerson gives us the max flow in time $O((m + 2m + n)m) = O(m^2 + mn)$ since $v_* \leq m$ and $G'$ has $m + 2m + n$ edges, which is polynomial.
\end{shaded}

\textbf{Bonus:} Prove that $G$ is $\mu$-orientable if and only if
$|F|\leq\mu(V(F))$ holds for all $F\subseteq E$, where $V(F)$ is
the set of endpoints of $F$ and $\mu(S)\defeq\sum_{v\in S}\mu(v)$.

\section*{Problem 2 {[}20 Points{]}: Vertex Disjoint Path}

Provide an $O(m\sqrt{n})$ time algorithm that given a strongly connected
$n$-node $m$-edge directed graph $G=(V,E)$ and vertices $s$ and
$t\in V$ with $s\neq t$ computes a set of vertex-disjoint (other
than $s$ and $t$) $s$-$t$ paths of maximum size (in terms of the
number of paths).
\begin{shaded}
    SOLUTION: Construct a new graph $G'$ as follows. For every interior vertex of $G$ (i.e. vertex besides $s,t$), let $v_{in} \neq v_{out}$ be new vertices in $V'$, and let $s = s_{in} = s_{out}$ and $t = t_{in} = t_{out} \in V'$. Construct edges $(v_{in}, v_{out}) \in E'$ of capacity $1$ for all $v \in V$ except $s,t$, and for all $e = (v, u) \in E$ construct an edge $(v_{out}, u_{in}) \in E'$ of capacity $1$.
    
    Any positive flow on $G'$ corresponds to $v_*(f)$ vertex-disjoint paths from $s$ to $t$ in the original graph. If $f$ is a flow on $G'$, we construct the paths on $G$ by letting $e = (v,u) \in G$ be part of the path if and only if $f_{(v_{out}, u_{in})} = 1$. There are a few things to check:
    \begin{itemize}
        \item \textit{These are valid $s$-$t$ paths:} Since the flow on $G'$ is positive, $f_{(s,v_{in})} = 1$ for at least one vertex $v$ which means there is an edge in these paths $(s,v)$ leaving $s$. Similarly there is at least one edge entering $t$. If an edge $e = (v,u)$ is part of the path then the flow into $u_{in}$ is $1$, which means the flow into $u_{out}$ is $1$, and so the flow out of $u_{out}$ is also $1$, i.e. there is another vertex $w$ such that $(u,w)$ is part of the path. Suppose there are two edges in the path $(v, u), (v, w)$ that are not vertex disjoint. Then $f_{(v_{out}, u_{in})} = 1$ and $f_{(v_{out}, w_{in})} = 1$ which means two units of flow are leaving $v_{out}$. But $v_{out}$ only has $1$ incoming vertex $v_{in}$ of capacity $1$, so this is a contradiction. The claim follows.
        \item \textit{There are $v_*(f)$ paths}: Note that the flow leaving $s$ in $G'$ is $v_*(f)$, which means there are this many edges $(s,v)$ in $G$ such that $f_{(s, v_{in})} = 1$. These are all chosen by the above correspondance so we see that we get $v_*(f)$ vertex-disjoint paths in this way. 
        \item \textit{Any $k$ vertex-disjoint $s$-$t$ paths in $G$ gives a flow $f$ on $G'$ with $v_*(f) = k$:} Suppose we have these paths on $G$. Then for every $(v,u)$ in the edges in the paths, send $1$ unit of flow along $(v_{out}, u_{in})$ and $1$ unit of flow along $(u_{in}, u_{out})$. This is clearly a valid flow since at the internal vertices $v$ of $G$, there is exactly one edge leaving $v$ and one entering $v$, which means $1$ unit of flow is sent into and out of $v_{in}$ and the same for $v_{out}$. Since there are $k$ edges $(s, v)$ in the paths leaving $s$, there are $k$ units of flow sent out of the edges $(s, v_{in})$ and so this flow clearly has value $k$.
    \end{itemize}
    It follows that the maximum number of vertex-disjoint paths in $G$ is exactly the max flow of $G'$. The vertex-disjoint paths in $G$ is a special case of an $s$-$t$ flow on $G$ if we give every edge unit capacity. Denote this flow in $G$ corresponding to a flow $f$ on $G'$ by $f_G$. 
    
    Then $G'_f = (G_{f_G})'$, i.e. the residual graph under $f$ of $G'$ equals the graph obtained by applying the construction from $G$ to $G'$ above to the residual graph $G_{f_G}$, albeit with different vertex and edge labelings. If $f$ sends $1$ unit of flow across $\ldots\to v_{out} \to u_{in} \to u_{out} \to w_{in} \to \ldots$ in $G'$, then in $G'_f$ these edges are reversed $\ldots\leftarrow v_{out} \leftarrow u_{in} \leftarrow u_{out} \leftarrow w_{in} \leftarrow \ldots$ with capacity $1$, and all edges with flow $0$ are unchanged. $G_{f_G}$ reverses the corresopnding edges $\ldots \leftarrow v \leftarrow u \leftarrow w \leftarrow \ldots$, and $G_{f_G}'$ separates these into in and out vertices. If we change the labeling the claim is obvious. 

    Then a flow on the residual graph $G'_f$ gives $v_*(G'_f)$ vertex-disjoint $s$-$t$ paths on $G_{f_G}$ by the above argument. Each path has at least $d_{f_G}(t)-1$ vertices besides $s$ and $t$, where $d_{f_G}(t)$ is the length in edges of the shortest path from $s$ to $t$ in $G_{f_G}$. Since the paths are vertex-disjoint, 
    \[v_*(G'_f)(d_{f_G}(t) - 1) \leq n\]
    and, since $d_f(t)/2 \leq d_{f_G}(t) - 1$ (we can construct an $s$-$t$ path in $G'_f$ of this distance by taking the shortest path in $G_{f_G}$ and adding the edges $v_{in}\to v_{out}$ or $v_{out}\to v_{in}$ for every intermediate $v$ in the path),
    \[v_*(G'_f) \leq \frac{2n}{d_{f}(t)}\]
    We can replace the bound in Lemma 7.8 in the notes with this, and apply it in Theorem 7.9 to show that running the blocking flow algorithm on $G'$ to compute $v_*(G')$ runs in time $O(m \sqrt{n})$. Roughly, every iteration runs in $O(m)$ so we need to show it terminates for some $i = O(\sqrt{n})$. Since $d_{f_i} \geq i$ for all $i$ in which the algorithm has not terminated, if $i \geq \lceil \sqrt{n} \rceil$ then $d_{f_i} \geq \sqrt{n}$. By the above bound, $v_*(G'_{f_i}) \leq 2\sqrt{n}$. Since the algorithm decreases $v_*(G'_{f_i})$ in every iteration, it must run for at most another $2 \sqrt{n}$ iterations, so it terminates in time $O(\sqrt{n})$ and the claim follows.
\end{shaded}

\section*{Problem 3 {[}10 Points{]}: Restricted Simple Path}

Provide an $O(m)$ time algorithm that given a connected undirected
graph $G=(V,E)$ with $m=|E|$ and three nodes, $u,v,w\in V$ outputs
a simple path from $u$ to $w$ passing through $v$ (provided one
exists).\\
\begin{shaded}
    SOLUTION: Construct a graph $G'$ as follows. Let $s,t$ be source and sink vertices, and let $u,v,w \in V'$. Then for every other vertex $a \in V$, split it into $a_{in}, a_{out} \in V'$ as before (and let $s = s_{in} = s_{out}$, ... $w = w_{in}= w_{out}$ for simplicity). Construct one edge $(s, v)$ of capacity $2$, and two edges $(u, t), (w, t)$ of capacity $1$ in $E'$. For every edge $e = \{a,b\} \in E$, construct edges $(a_{out},b_{in}), (b_{out},a_{in}) \in E'$ of capacity $1$, and construct edges $(a_{in}, a_{out})$ of capacity $1$ for every $a \in V$ besides $s,t,u,v,w$.

    Then $v_*(G') = 2$ if and only if there is a simple path from $u$ to $w$ through $v$ in $G$. Suppose $v_*(G') = 2$. Then two units of flow go from $s$ to $v$, and $1$ unit goes from each of $u,w$ to $t$. For every edge $(a_{out},b_{in})$ in $G'$ with flow $1$, add $\{a,b\}$ to the path in $G$. By the same reasoning as in Problem 2, all edges with positive flow in $G'$ have to be vertex disjoint in $G$ since otherwise we could for example have two units of flow leaving $a_{out}$, but only one incoming edge $a_{in} \to a_{out}$ of capacity $1$. For this to be a valid flow it is obvious that there must be a path in $G'$ of flow $1$ from $v$ to both $u$ and $w$, which means the edges in $G$ at least connect $v$ to $u$ and $w$. If we reverse the path from $v$ to $u$ in $G$, we get a path from $u$ to $w$ through $v$.

    Now suppose we have a simple path from $u$ to $w$ through $v$. We traverse the paths in order from $u$ to $v$, i.e. if the first edge is $\{u, a^1\}$ send flow along $(a^1_{out},u)$ and $(a^1_{in}, a^1_{out})$. If the next edge is $\{a^1, a^2\}$ where $a^1$ was incident to the previous edge in the path, send flow along $(a^2_{out}, a^1_{in})$ and $(a^2_{in}, a^2_{out})$. Continue until we reach $v$, i.e. we send flow along $(v, a^k_{in})$ (but not from $v$ to itself). The next edge is also incident to $v$, $\{v, b^1\}$ so send flow along $(v, b^1_{in})$ and $(b^1_{in}, b^1_{out})$. If the next edge is $\{b^1, b^2\}$ where $b^1$ was incident to the previous edge, send flow along $(b^1_{out}, b^2_{in})$ and $(b^2_{in}, b^2_{out})$. Continue until we send flow along $(b^j_{out}, w)$ (but not $w$ to itself). Lastly, send two units of flow along $(s, v)$ and $1$ unit along each of $(u, t), (w, t)$. It is apparent that this is an $s$-$t$ flow of value $2$ since the vertices in the path in $G$ are disjoint, implying we never sent more than $1$ unit of flow along the interior directed edges $(a_{in}, a_{out})$. 

    By Ford-Fulkerson, since $v_*(G') \leq 2$, computing max flow on this graph is possible in time 
    \[O(|E'|v_*(G')) = O((2m + (n - 3) + 3)2) = O(m + n) = O(m)\] 
    where we used $n - 1 \leq m$ as $G$ is connected and any spanning tree has $n-1$ edges.
\end{shaded}
\textbf{\emph{Hint}}\emph{: It may be helpful to solve parts of problem
2 first}.
\end{document}
